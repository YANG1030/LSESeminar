\usepackage[utf8]{inputenc}
%%%%%%%%% Theorem Environment

%%% Use mdframed

\usepackage[framemethod=TikZ]{mdframed}
\usepackage{amsthm}

\usepackage{fontawesome}
\usepackage{bbding}

% \mdfsetup{skipabove=\topskip,skipbelow=\topskip}
% \mdfdefinestyle{theoremstyle}{%
%   outerlinewidth=5pt,innerlinewidth=0pt,
%   outerlinecolor=gray!40,roundcorner=5pt
% }

%%% Proof

\renewcommand*{\proofname}{\blue{\textbf{Proof.}}}
% \renewcommand*{\qedsymbol}{$\mathwitch*$}
% \renewcommand*{\qedsymbol}{\faCoffee}
\renewcommand*{\qedsymbol}{\Peace}

%%% Theorem
\definecolor{Mingu-Red}{RGB}{163, 25, 25}
\mdfdefinestyle{theoremstyle}{%
    middlelinewidth=2pt,
    middlelinecolor=Mingu-Red,
    %%%%%%%%%
    linecolor=gray,linewidth=.5pt,%
    backgroundcolor=gray!10,
    %%%%%%%%%
    skipabove=8pt,
    skipbelow=5pt,
    %%%%%%%%%
    innertopmargin=.4\baselineskip,
    innerbottommargin=.4\baselineskip,
    innerleftmargin=7pt,
    innerrightmargin=7pt,
    %%%%%%%%%
    frametitlerule=true,%
    frametitlerulewidth=.5pt,
    frametitlebackgroundcolor=gray!15,
    frametitleaboveskip=0pt,
    frametitlebelowskip=0pt,
    %%%%%%
    % shadow=true,shadowsize=3pt,shadowcolor=black!20,
    theoremseparator={:},
    roundcorner=5pt,
}

\mdtheorem[style=theoremstyle]{exercise}{\faThumbTack\; Exercise}[section]
\mdtheorem[style=theoremstyle]{MDtheorem}{\faGraduationCap\; Theorem}[section]
\mdtheorem[style=theoremstyle]{proposition}{\faGraduationCap\; Proposition}[section]
\mdtheorem[style=theoremstyle]{corollary}{\faThumbTack\; Corollary}[section]
\mdtheorem[style=theoremstyle]{lemma}{\faThumbTack\; Lemma}[section]
\mdtheorem[style=theoremstyle]{problem}{\faQuestion\; Problem}[section]

%%% 定义theorem environment, 给theorem title加括号
\newcommand*{\Title}{}
\newenvironment{theorem}[1][]{%
    \ifstrempty{#1}{\begin{MDtheorem}}{\begin{MDtheorem}[(#1)]}%
}{%
    \end{MDtheorem}%
}%
%%%
\usepackage{lipsum}

\newtheoremstyle{examplestyle}% ⟨name⟩
{3pt}%⟨Space above⟩
{3pt}%⟨Space below⟩
{\upshape}%⟨Body font⟩
{}%⟨Indent amount⟩
{\bfseries}% ⟨Theorem head font⟩
{.}%⟨Punctuation after theorem head⟩
{.5em}%⟨Space after theorem head⟩2
{}%⟨Theorem head spec (can be left empty, meaning ‘normal’)⟩

\theoremstyle{examplestyle}
\newtheorem{myexam}{\HandPencilLeft\; Example}
\mdfsetup{%
    middlelinecolor=black,
    middlelinewidth=0.5pt,
    backgroundcolor=white,
    % roundcorner=5pt
}
\newenvironment{example}
  {\begin{mdframed}\begin{myexam}}
  {\end{myexam}\end{mdframed}}



\mdfdefinestyle{definitionstyle}{%
  middlelinewidth=2pt,
  middlelinecolor=Mingu-Red,
  %%%%%%%%%
  linecolor=gray,linewidth=.5pt,%
  backgroundcolor=white,
  %%%%%%%%%
  skipabove=8pt,
  skipbelow=5pt,
  %%%%%%%%%
  innertopmargin=.4\baselineskip,
  innerbottommargin=.4\baselineskip,
  innerleftmargin=7pt,
  innerrightmargin=7pt,
  %%%%%%%%%
  frametitlerule=true,%
  frametitlerulewidth=.5pt,
  frametitlebackgroundcolor=gray!15,
  frametitleaboveskip=0pt,
  frametitlebelowskip=0pt,
  %%%%%%
%   shadow=true,shadowsize=3pt,shadowcolor=black!20,
  theoremseparator={:},
  roundcorner=5pt
}

% \mdtheorem[style=lemmastyle]{remark}{$\blacklozenge$ Remark}[section]
\mdtheorem[style=definitionstyle]{definition}{\faKey\; Definition}[section]

\newtheoremstyle{remarkstyle}% ⟨name⟩
{3pt}%⟨Space above⟩
{3pt}%⟨Space below⟩
{\upshape}%⟨Body font⟩
{}%⟨Indent amount⟩
{\bfseries}% ⟨Theorem head font⟩
{.}%⟨Punctuation after theorem head⟩
{.5em}%⟨Space after theorem head⟩2
{}%⟨Theorem head spec (can be left empty, meaning ‘normal’)⟩

\theoremstyle{remarkstyle}
\newtheorem{remarkx}{\HandRight\; Remark}
\newenvironment{remark}
  {\pushQED{\qed}\renewcommand{\qedsymbol}{\PencilLeftDown}\remarkx} % 给 remark 环境定义结束符
  {\popQED\endremarkx}

% \definecolor{remarkcolor-bk}{RGB}{255,242,242}
% \definecolor{remarkcolor-line}{RGB}{0,242,12}
% \theoremstyle{remarkstyle}
% \newtheorem*{myremark}{Remark}
% \mdfsetup{%
%     middlelinecolor=remarkcolor-line,
%     middlelinewidth=1pt,
%     backgroundcolor=remarkcolor-bk,
%     roundcorner=5pt}
% \newenvironment{remark}
%   {\begin{mdframed}\begin{myremark}}
%   {\end{myremark}\end{mdframed}}



%%%%%%%%%%%%%%%%%%%%%%%%%%%%%%%%%%%%%%%%%%%%%%%%%%%%%%%%%%%%%%%%%%%%%%%%%%%%%%%%%%%%%%%%%%%%%%%%%%%% Use package amsthm

% \usepackage{amsthm}
% \theoremstyle{plain}
% \newtheorem{theo}{Theorem}
% \newtheorem{prop}{Proposition}
% \newtheorem{lem}{Lemma}
% \newtheorem{coro}{Corollary}

% \newtheoremstyle{definition}% ⟨name⟩
% {3pt}%⟨Space above⟩
% {3pt}%⟨Space below⟩
% {\itshape}%⟨Body font⟩
% {}%⟨Indent amount⟩
% {\bfseries}% ⟨Theorem head font⟩
% {.}%⟨Punctuation after theorem head⟩
% {.5em}%⟨Space after theorem head⟩2
% {}%⟨Theorem head spec (can be left empty, meaning ‘normal’)⟩

% \theoremstyle{definition}
% \newtheorem{exam}{Example}
% \newtheorem*{sol*}{Solution}
% \newtheorem{exer}{Exercise}
% \newtheorem*{ans*}{Answer}
% \newtheorem{defi}{Definition}

% \newtheoremstyle{note}% style name
% {2ex}% above space
% {2ex}% below space
% {\small\itshape}% body font
% {}% indent amount
% {\normalfont\bfseries}% head font
% {.}% post head punctuation
% {1ex}% post head space
% {}% head spec
% \theoremstyle{note}
% \newtheorem{remark}{Remark}
% \newtheorem*{hint*}{Hint}



%Font
\usepackage[T1]{fontenc}
\usepackage{mathpazo}
%\usepackage{ebgaramond}
\usepackage{tgpagella}
\usepackage{xcolor}
\usepackage{amsmath}
\usepackage[colorlinks,linkcolor=blue]{hyperref}
\usepackage{IEEEtrantools}
\usepackage{dsfont}


\usepackage{enumitem}%列表环境
\usepackage[square]{natbib}
\usepackage{graphicx}
\usepackage{subfigure}  
\usepackage{tikz}
\usepackage{tikz-cd}
\usetikzlibrary{arrows.meta}
\usepackage[paper=a4paper,left=30mm,right=20mm,top=25mm,bottom=30mm]{geometry}

\usepackage{arydshln}

\usepackage{marginnote}
\usepackage{geometry}
\geometry{a4paper,inner=2cm,outer=2cm,top=2.5cm,bottom=2.5cm,marginparwidth=1.5cm, marginparsep=0.5cm}

%圆圈数字
\usepackage{tikz}
\newcommand*{\circled}[1]{\lower.7ex\hbox{\tikz\draw (0pt, 0pt)%
		circle (.5em) node {\makebox[1em][c]{\small #1}};}}

%行距
\usepackage{setspace}
\renewcommand{\baselinestretch}{1.2} 

\newcommand*{\colorboxed}{}
\def\colorboxed#1#{%
	\colorboxedAux{#1}%
}
\newcommand*{\colorboxedAux}[3]{%
	% #1: optional argument for color model
	% #2: color specification
	% #3: formula
	\begingroup
	\colorlet{cb@saved}{.}%
	\color#1{#2}%
	\boxed{%
		\color{cb@saved}%
		#3%
	}%
	\endgroup
}

\usepackage{fancyhdr} 
\pagestyle{fancy}
\newlength\mylen
\addtolength\mylen{\marginparsep}
\addtolength\mylen{\marginparwidth}
%\fancyheadoffset[OR,EL]{\the\mylen}%把headrule延长至右边
%\fancyfootoffset[OR,EL]{\the\mylen}%把footrule延长至右边
\usepackage{lastpage}
\fancyhead[RE,LO]{\thepage \;of \,\pageref{LastPage}}
\fancyhead[LE,RO]{\textsf{Statistics Seminar (2021 Fall, LSE)}}
\fancyfoot[LE,RO]{ROUGH DRAFT. DO NOT DISTRIBUTE!}
\fancyfoot[RE,LO]{\textsf{Latest Updated: \today}}
\cfoot{}
\renewcommand{\headrulewidth}{0.4pt}
\renewcommand{\footrulewidth}{0.4pt}

%Important symbol
\usepackage{tabto}
\newcommand\importantsymbol[1][0pt]{%
	\tabto*{0cm}\makebox[\dimexpr-0.3cm-#1\relax][r]{$\blacktriangleright$}\tabto*{\TabPrevPos}}
\newcommand\veryimportantsymbol[1][0pt]{%
	\tabto*{0cm}\makebox[\dimexpr-0.3cm-#1\relax][r]{$\blacktriangleright\blacktriangleright$}\tabto*{\TabPrevPos}}
\newcommand\veryveryimportantsymbol[1][0pt]{%
	\tabto*{0cm}\makebox[\dimexpr-0.3cm-#1\relax][r]{$\blacktriangleright\blacktriangleright\blacktriangleright$}\tabto*{\TabPrevPos}}
%Question symbol
\newcommand\questionsymbol[1][0pt]{%
	\tabto*{0cm}\makebox[\dimexpr-0.3cm-#1\relax][r]{?}\tabto*{\TabPrevPos}}
\newcommand\indep{\protect\mathpalette{\protect\independenT}{\perp}}
\def\independenT#1#2{\mathrel{\rlap{$#1#2$}\mkern2mu{#1#2}}}


\usepackage{amsmath,amssymb,amsfonts,amssymb,amsthm}
\numberwithin{equation}{section}%公式编号与章节关联
\usepackage{dsfont}%示性函数
\allowdisplaybreaks[4]%长公式换页
\usepackage{bbm}
\usepackage[us,24hr]{datetime}
\usepackage{extarrows}%\xlongequal
\usepackage{IEEEtrantools}
\usepackage{mathrsfs}%花体字母
\usepackage{multicol}%分栏
\usepackage{booktabs}%使三线表首尾线更粗
\usepackage{hyperref}
\hypersetup{
	colorlinks=true,
	linkcolor=blue,
	filecolor=magenta,      
	urlcolor=cyan,
}

\newcommand{\vkl}{\vskip 0.10in}
\newcommand{\vkL}{\vskip 0.20in}
\newcommand{\vkU}{\vskip 0.30in}

%%%%%%%%%%%%%%%%%%%%%%%%%%%%%%%%%%%%%自定义符号%%%%%%%%%%%%%%%%%%%%%%%%%%%%%%%%%%%%%%%%%%%%%%%%%%%

%%%%%%大写黑体%%%%%%
\newcommand{\bA}{\mathbf{A}}
\newcommand{\bB}{\mathbf{B}}
\newcommand{\bC}{\mathbf{C}}
\newcommand{\bD}{\mathbf{D}}
\newcommand{\bE}{\mathbf{E}}
\newcommand{\bF}{\mathbf{F}}
\newcommand{\bG}{\mathbf{G}}
\newcommand{\bH}{\mathbf{H}}
\newcommand{\bI}{\mathbf{I}}
\newcommand{\bL}{\mathbf{L}}
\newcommand{\bO}{\mathbf{O}}
\newcommand{\bP}{\mathbf{P}}
\newcommand{\bM}{\mathbf{M}}
\newcommand{\bN}{\mathbf{N}}
\newcommand{\bR}{\mathbf{R}}
\newcommand{\bS}{\mathbf{S}}
\newcommand{\bT}{\mathbf{T}}
\newcommand{\bU}{\mathbf{U}}
\newcommand{\bV}{\mathbf{V}}
\newcommand{\bW}{\mathbf{W}}
\newcommand{\bX}{\mathbf{X}}
%%%%%%小写黑体%%%%%%
\newcommand{\ba}{\mathbf{a}}
\newcommand{\bb}{\mathbf{b}}
\newcommand{\bc}{\mathbf{c}}
\newcommand{\bd}{\mathbf{d}}
\newcommand{\be}{\mathbf{e}}
\newcommand{\bi}{\mathbf{i}}
\newcommand{\bj}{\mathbf{j}}
\newcommand{\bp}{\mathbf{p}}
\newcommand{\bu}{\mathbf{u}}
\newcommand{\bw}{\mathbf{w}}
\newcommand{\bv}{\mathbf{v}}
\newcommand{\bx}{\mathbf{x}}
\newcommand{\by}{\mathbf{y}}
\newcommand{\bz}{\mathbf{z}}
%%%%%%空心%%%%%%
\newcommand{\bbR}{\mathbb{R}}
\newcommand{\bbQ}{\mathbb{Q}}
\newcommand{\bbZ}{\mathbb{Z}}
\newcommand{\bbC}{\mathbb{C}}
%%%%%%花体%%%%%%
\newcommand{\calA}{\mathcal{A}}
\newcommand{\calB}{\mathcal{B}}
\newcommand{\calC}{\mathcal{C}}
\newcommand{\calD}{\mathcal{D}}
\newcommand{\calM}{\mathcal{M}}
\newcommand{\calN}{\mathcal{N}}
\newcommand{\calT}{\mathcal{T}}
\newcommand{\calV}{\mathcal{V}}
%%%%%%符号%%%%%%
\newcommand{\Prob}{\mathbb{P}}
\newcommand{\Expe}{\mathsf{E}\,}
\newcommand{\Var}{\mathsf{Var}}
\newcommand{\Cov}{\mathsf{Cov}}
\newcommand{\rank}{\mathsf{rank}}
\newcommand{\tr}{\mathsf{tr}}
\newcommand{\sign}{\mathsf{sign}}
\newcommand{\diag}{\mathsf{diag}}
\newcommand{\res}{\mathsf{Res}}
\newcommand{\spt}{\mathrm{spt}}
\newcommand\independent{\protect\mathpalette{\protect\independenT}{\perp}}
%\renewcommand{\Re}{\mathsf{Re}}
%\renewcommand{\Im}{\mathsf{Im}}
\newcommand{\st}{\mathrm{s.t.}}
\newcommand{\io}{\mathrm{i.o.}}
\newcommand{\ult}{\mathrm{ult.}}
\newcommand{\md}{\mathrm{d}}
\newcommand*{\dif}{\mathop{}\!\mathrm{d}}%微分算子
\newcommand{\norm}[1]{\left\lVert#1\right\rVert}%norm
\newcommand{\wc}{\overset{\mathsf{w}}{\longrightarrow}}
%\newcommand{\<}{\langle}
%\newcommand{\>}{\rangle}
\newcommand{\overbar}[1]{\mkern 1.5mu\overline{\mkern-1.5mu#1\mkern-1.5mu}\mkern 1.5mu}
\newcommand{\He}{\mathsf{H}}
\DeclareMathOperator{\as}{a.s.}
\renewcommand{\bar}{\overline}
\newcommand{\Range}{\mathrm{Range}}
\DeclareMathOperator{\spn}{span}
\newcommand{\indicator}{\mathds{1}}
%inner product notation:	
%Ref: https://tex.stackexchange.com/questions/309026/how-to-define-an-inner-product-argument-in-latex
\usepackage{mathtools}
\DeclarePairedDelimiterX{\inp}[2]{\langle}{\rangle}{#1, #2}%inner product
%augmented matrix
\makeatletter
\renewcommand*\env@matrix[1][*\c@MaxMatrixCols c]{%
	\hskip -\arraycolsep
	\let\@ifnextchar\new@ifnextchar
	\array{#1}}
\makeatother
%%%%%%颜色%%%%%%
\newcommand{\red}[1]{\textcolor{red}{#1}}
\newcommand{\blue}[1]{\textcolor{blue}{#1}}
%%%%%%%%%%%%%%%%%%%%%%%%%%%%%%%%%%%%%%%%%%%%%%%%%%%%%%%%%%%%%%%%%%%%%%%%%%%%%%%%%%%%%%%%%%%%%%%%%%%%%%%%

%%%%%%%%%%%%%%%%%%%%%%%%%%%%%%%%%%%%%定理环境%%%%%%%%%%%%%%%%%%%%%%%%%%%%%%%%%%%%%%%%%%%%%%%%%%%%%%%%%%%%
\theoremstyle{definition}
\newtheorem{exmp}{Example}[section]
%\newtheorem{theorem}{Theorem}[section]
%\newtheorem{corollary}[theorem]{Corollary}
%\newtheorem{lemma}[theorem]{Lemma}
%\newtheorem{proposition}[theorem]{Proposition}
%\newtheorem{remark}[theorem]{Remark}
%\newtheorem{definition}[theorem]{Definition}
%\newtheorem{property}[theorem]{Property}
\let\emph\relax % there's no \RedeclareTextFontCommand
\DeclareTextFontCommand{\emph}{\bfseries\em}

\newenvironment{dedication}
{\clearpage           % we want a new page
	\thispagestyle{empty}% no header and footer
	\vspace*{\stretch{1}}% some space at the top 
	\itshape             % the text is in italics
	%\raggedleft          % flush to the right margin
	\centering
}
{\par % end the paragraph
	\vspace{\stretch{3}} % space at bottom is three times that at the top
	\clearpage           % finish off the page
}
%%%%%%%%%%%%%%%%%%%%%%%%%%%%%%%%%%%%%%%%%%%%%%%%%%%%%%%%%%%%%%%%%%%%%%%%%%%%%%%%%%%%%%%%%%%%%%%%%%%%%%%%

%%%%%%%%%%%%%%%%%%%%%%%%%%%%%%%%%%%%%%%%%%Title Page%%%%%%%%%%%%%%%%%%%%%%%%%%%%%%%%%%%%%%%%%%%%%%%%%%%%
\newcommand{\HRule}[1]{\rule{\linewidth}{#1}} 	% Horizontal rule

\makeatletter							% Title
\def\printtitle{%						
	{\centering \@title\par}}
\makeatother									

\title{	
	\normalsize \textit{London School of Economics and Political Science}	% Subtitle
	\\[2.0cm]								% 2cm spacing
	\HRule{1pt} \\						% Upper rule
	\LARGE {{\bfseries\itshape Lecture Notes for Statistics Seminar}\\ 
		\normalsize{\scshape 2021 Fall}}	% Title
	\HRule{2pt} \\ [1cm]		% Lower rule + 0.5cm spacing
	\normalsize(Scribed and Edited by {\scshape YANG Xuzhi})
}
%%%%%%%%%%%%%%%%%%%%%%%%%%%%%%%%%%%%%%%%%%%%%%%%%%%%%%%%%%%%%%%%%%%%%%%%%%%%%%%%%%%%%%%%%%%%%%%%%%%%%%%%

%%%%%%%%%%%%%%%%%%%%%%%%%%%%%%%%%%%%%%%%%%Chapter Format%%%%%%%%%%%%%%%%%%%%%%%%%%%%%%%%%%%%%%%%%%%%%%%%%%%%
\usepackage{titlesec}

\titleformat
{\chapter} % command
[display] % shape
{\bfseries\LARGE\itshape} % format
{Lecture  \thechapter} % label
{0.5ex} % sep
{
	\rule{\textwidth}{1pt}
	\vspace{1ex}
	\centering
} % before-code
[
\vspace{-0.5ex}%
\rule{\textwidth}{0.3pt}
] % after-code
%%%%%%%%%%%%%%%%%%%%%%%%%%%%%%%%%%%%%%%%%%%%%%%%%%%%%%%%%%%%%%%%%%%%%%%%%%%%%%%%%%%%%%%%%%%%%%%%%%%%%%%%

\date{\textsc{Lastest Updated}:\quad\today,\; \currenttime}
