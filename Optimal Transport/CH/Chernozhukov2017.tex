\chapter{Monge-Kantorovich Depth, Quantiles, Ranks and Signs (Chernozhukov et al. 2017)}
\section{Motivation}
\faQuoteLeft  An important feature of halfspace depth is the convexity of its contours, which thus satisfy the star-convexity requirement embodied in the linear monotonicity axiom. That feature is shared by most existing depth concepts and might be considered undesirable for distributions with non-convex supports or level contours, and multi-modal ones \faQuoteRight 

\section{Idea}
\textbf{Case 1: Univariate Distribution Family $\mathcal{P}^1$.}

The first case is that of the family $\mathcal{P}^1$ of all univariate distributions with non-vanishing Lebesgue densities. Here, the definition of quantile and distribution function, ranks and signs are the "\textcolor{magenta}{classical}" univariate ones.

\noindent\textbf{Case 2: Elliptical Distribution Family $\mathcal{P}_{ell}^d$ over $\mathbb{R}^d$.}

Let $X \sim P_{\mu, \Sigma, g} \in \mathcal{P}_{ell}^d$, where $\mu$ and $\Sigma$ are location parameter and scatter parameter respectively, and $g$ is radial density (with radial distribution $G$). Then the following are equivalent:
\[
X \sim P_{\mu, \Sigma, g} \iff Y \triangleq \Sigma^{-1/2}(X - \mu) \sim P_{0, I, g} \iff R_P(Y) \triangleq \frac{Y}{||Y||}G(||Y||) \sim U_d,
\]
where $U_d$ denotes the spherical uniform distribution (RP: $U_d = r\phi$, $r \independent \phi$). 
\begin{itemize}
	\item Note that the spherical contours with $P_{\mu, I, g}$-probability contents $\tau$ coincide with the halfspace depth contours. 
	\item Thus we can naturally define a \textit{$\tau$-quantile contour} as the contour of the depth region with probability content at least $\tau$ (contour of equal depth of $\tau$). 
	\item $R_p(Y)$, $R_p(Y)/||R_p(Y)||$ and $||R_p(Y)||$ play the roles of \textit{vector ranks, signs and ranks}, respectively. 
	\item The inverse map $u \longrightarrow Q_P(u)$ of the vector rank map $y \longrightarrow R_P(y)$ is called \textit{vector quantile} map 
\end{itemize}

\begin{remark}
	Take a closer look at the case, we can observe that the \textit{$\tau$-quantile contour} (contour of the depth region with probability content $\tau$) is $Q_P(\mathcal{S}(\tau))$, while \textit{the depth region with probability content} $\tau$ is $Q_P(\mathbb{S}(\tau))$.
\end{remark}

\noindent\textbf{\textcolor{magenta}{Case 3: General Distribution $P$ over $\mathbb{R}^d$}.}

From the insights we gained through the traditional cases, for general distribution $P$ over $\mathbb{R}^d$, we also want to define a map $Q_P$ that transform the spherical uniform distribution $U_d$ into the target distribution. The following theorem due to \cite{brenier1991polar} and \cite{mccann1995existence} guarantees the existence of such map.

\section{Definition of MK Depth, Quantiles, Ranks and Signs}
\subsection{Step 1: Find an optimal transport}
\begin{theorem}\label{th: existence}
	Let $P$ and $F$ be two distributions on $\mathbb{R}^d$.
	\begin{enumerate}
		\item[(1)] If $F$ is absolutely continous with respect to the Lebesgue measure on $\mathbb{R}^d$, the following holds: for any convex set $\mathcal{U} \subset \mathbb{R}^d$ containing the support of $F$, there exists a convex function $\psi: \mathcal{U} \longrightarrow \mathbb{R}\cup\{+\infty\}$ such that $\nabla\psi \# F = P$. The gradient $\nabla\psi$ of that function exists and unique, $F$-a.e.
		\item[(2)]If, in addition, $P$ is absolutely continous on $\mathbb{R}^d$ the following holds: for any convex set $\mathcal{Y} \subset \mathbb{R}^d$ containing the support of $P$, there exists a convex function $\psi^*: \mathcal{Y} \longrightarrow \mathbb{R}\cup\{+\infty\}$ such that $\nabla\psi^* \# P = F$. The gradient $\nabla\psi^*$ of $\psi^*$ exists and unique, and $\nabla\psi^* = \nabla\psi^{-1}$, $P$-a.e.
	\end{enumerate}
\end{theorem}
\begin{remark}
	Note that although $\psi$ and $\psi^*$ may not be unique, the map $\nabla\psi$ and $\nabla\psi^*$ is unique a.e.
\end{remark}
We will see that if $P$ and $F$ have \textcolor{magenta}{finite second moments}, then $\nabla\psi$ is an \textcolor{magenta}{optimal transport plan} from $F$ to $P$ under \textit{quadratic cost}. That is, 
%\begin{IEEEeqnarray}{rCl}
%	\nabla\psi & = & \arg\inf_{\pi \in \Pi(F, P)}\int ||x - y||^2 \md \pi(x, y) \label{eq: QuadOP}
%\end{IEEEeqnarray}
\begin{IEEEeqnarray}{rCl}
	\nabla\psi & = & \arg\inf_{Q \# F = P}\int (u - Q(u))^2 \md F(y) \label{eq: QuadOP}
\end{IEEEeqnarray}

The following proposition will help us to establish a dual problem of (\ref{eq: QuadOP}).
\begin{proposition}
	If $\mu$ and $\nu$ be two probability measures on $\mathbb{R}^d$ with \textcolor{magenta}{finite seond moments}, then (\ref{eq: kantDual}) with quadratic cost becomes
	\begin{IEEEeqnarray}{rCl}
		\sup_{ \Pi(\mu, \nu)}\int (x \cdot y) \md \pi(x, y) = \inf_{\tilde{\Phi}}J(\phi, \psi), \label{eq: quaddual}
	\end{IEEEeqnarray}
where $\tilde{\Phi} = \{(\phi, \psi) \in L^1(\md\mu) \times L^1(\md\nu): x \cdot y \leq \phi(x) + \psi(y) \ a.e. \}$.
\end{proposition}
\begin{proof}
	From the definition of $\Phi_C$, we have for any $(\phi, \psi) \in \Phi_C$,
	\begin{IEEEeqnarray}{rCl}
		\phi(x) + \psi(y) \leq \frac{|x - y|^2}{2}, \nonumber
	\end{IEEEeqnarray}
	this can be rewritten as
	\begin{IEEEeqnarray}{rCl}
		x \cdot y \leq\left[\frac{|x|^{2}}{2}-\phi(x)\right]+\left[\frac{|y|^{2}}{2}-\psi(y)\right] \nonumber.
	\end{IEEEeqnarray}
	If we define
	\begin{IEEEeqnarray}{rCl}
		\widetilde{\phi}(x)=\frac{|x|^{2}}{2}-\phi(x), \quad \widetilde{\psi}(y)=\frac{|y|^{2}}{2}-\psi(y) \nonumber,
	\end{IEEEeqnarray}
	and forget about the $\sim$ symbol, then we can see
	\begin{IEEEeqnarray}{rCl}
		\inf_{\Pi(\mu, \nu)}I[\pi] & = & M_{2} -\sup_{\Pi(\mu, \nu)}\int (x \cdot y) d \pi(x, y) \label{eq: quaddual1} \\
		\sup_{\Phi_c}J(\phi, \psi) & = & M_2 - \inf_{\tilde{\Phi}}J(\phi, \psi) \label{eq: quaddual2},
	\end{IEEEeqnarray}
	where $M_{2}\triangleq\int_{\mathbb{R}^{n}} \frac{|x|^{2}}{2} d \mu(x)+\int_{\mathbb{R}^{n}} \frac{|y|^{2}}{2} d \nu(y)<+\infty$. Thus from (\ref{eq: quaddual1}) and (\ref{eq: quaddual2}) we obtained (\ref{eq: quaddual}).
\end{proof}

\begin{definition}
	For any function $\psi: \mathcal{U} \longrightarrow \mathbb{R}\bigcup\{+\infty\}$, the \textit{\textcolor{magenta}{conjugate}} $\psi^*: \mathcal{Y} \longrightarrow \mathbb{R}\bigcup\{+\infty\}$ of $\psi$ is defined as
	\begin{IEEEeqnarray}{rCl}
		\psi^{*}(y)&\triangleq&\sup _{u \in \mathcal{U}}\left[u \cdot y-\psi(u)\right], \ y \in \mathcal{Y} \label{eq: Legendre}.
	\end{IEEEeqnarray}
	Call $(\psi, \psi^*)$ \textit{\textcolor{magenta}{conjugate pair of patentials}} over $(\mathcal{U}, \mathcal{Y})$.
\end{definition}
\begin{remark}
	The transformation in (\ref{eq: Legendre}) is called \textcolor{magenta}{\bf Legendre Transformation}. From the definition we can see that any $(\psi, \psi^*) \in \tilde{\Phi}$. It can be proved that if $\psi$ is differential and strict convex, we have 
	\begin{IEEEeqnarray}{rCl}
		\nabla\psi^* = (\nabla\psi)^{-1} \label{eq: inverse},
	\end{IEEEeqnarray}
	the proof can be found at the Wikipedia page of Legendre Transformation. 
\end{remark}

The following theorem states that the infimum of $J$ on $\tilde{\Phi}$ is unchanged if one restricts $J$ to the very small subset of $\tilde{\Phi}$ which is made of all the conjugate pairs.

\begin{theorem}\label{th: dualOP}
	Let $\mu$, $\nu$ be two "proper" probability measures on $\mathbb{R}^n$, with finite second moments.
	\begin{itemize}
		\item[(i)] Then there exists a pair $(\psi, \psi^*)$ of lower semi-continuous propoer conjugate convex function on $\mathbb{R}^n$ such that 
		\begin{IEEEeqnarray}{rCl}
			\inf_{\tilde{\Phi}}J & = & J(\psi, \psi^*); \label{eq: psi}
		\end{IEEEeqnarray}
		\item[(ii)] $\nabla\psi$ and $\nabla\psi^*$ are the unique solutions of the Monge problem for transport $\mu$ to $\nu$ and transport $\nu$ to $\mu$ with a quadratic cost function, respectively;
		\item[(iii)] according to (\ref{eq: inverse}), for $\md \mu$-almost all $x$ and $\md \nu$-almost all $y$,
		\begin{IEEEeqnarray}{rCl}
			\nabla \psi^* \circ \nabla \psi (x) & = & x \nonumber \\
			\nabla \psi \circ \nabla \psi^* (y) & = & y \nonumber
		\end{IEEEeqnarray} 
	\end{itemize}
\end{theorem}
\begin{remark}
	From this theorem, we know 
	\begin{itemize}
		\item[(i)] where do the $\nabla\psi$ and $\nabla\psi^*$ in Theorem \ref{th: existence} come from and how to calculate them;
		\item[(ii)] if $P$ and $F$ in Theorem \ref{th: existence} have finite second moment condition, then $\nabla\psi$ and $\nabla\psi^*$ are optimal transport under quadratic cost function. 
		\item[(iii)] why the gradient $\nabla\psi^*$ is the inverse of $\nabla \psi$.
	\end{itemize}
\end{remark}
%
%The following theorem, refer to \cite{villani2021topics}, proves that the $\nabla\psi$ in the Theorem \ref{th: dualOP} is the unique optimal transport map from $F$ to $P$ for quadratic cost.
%
%\begin{theorem}
%	Suppose hypothesis (1) of Theorem \ref{th: existence} holds and $P$ and $F$ have finit second moments: then the conjugate pair of potentials $(\nabla\psi, \nabla\psi^*)$ solves the optimization problem $\inf_{\tilde{\Phi}}J$ in (\ref{eq: quaddual}).
%\end{theorem}

\subsection{Step 2: Define MK vector quantiels and vector ranks}
To define a multivariate rank on the interest distribution $P$ based on the reference distribution $F$, we only need to find the unique transport $R_P$ from $P$ to $F$, and the corresponding inverse transport $Q_P$ will naturally lead to the definition of quantile function.
\begin{definition}
	Denote by $\nabla\psi$ the $F$-almost surely unique gradient of a convex function $\psi$ in (1) of Theorem \ref{th: existence}, and let $\psi^*$ be the conjugate of $\psi$. Then we define \textit{\textcolor{magenta}{vector quantile}} $Q_P$ and \textit{\textcolor{magenta}{vector ranks}} $R_P$ as 
	\begin{IEEEeqnarray}{rCl}
			\mathrm{Q}_{P}(u) &\in& \arg \sup _{y \in \mathcal{Y}}\left[y^{\top} u-\psi^{*}(y)\right], \quad u \in \mathcal{U} \nonumber \\
			\mathrm{R}_{P}(y) &\in& \arg \sup _{u \in \mathcal{U}}\left[y^{\top} u-\psi(u)\right], \quad y \in \mathbb{R}^{d} \nonumber
	\end{IEEEeqnarray}
\end{definition}
\begin{remark}
	By the evelope theorem and Rademacher's theorem (see \cite{villani2021topics}), we have 
	\begin{align}
			\mathrm{Q}_{P}=\nabla \psi  \text { a.e. on } \mathcal{U}, \quad \mathrm{R}_{P}=\nabla \psi^{*}  \text { a.e. on } \mathcal{Y} \notag,
	\end{align}
	and the under some regularity conditions the "a.e." can become "for all". 
\end{remark}

\subsection{Step 3: Define MK depth, quantiles, rankds and signs}
Let's take $F = U_d$ and $P$ be an arbitriary distribution to deliver the multivariate notion of quantiles and ranks through $R_P$ and $Q_P$.
\begin{definition}
	\begin{enumerate}
		\item[(1)] The MK rank of $y \in \mathbb{R}^d$ is $||R_P(y)||$ and the MK sign is $R_P(y)/||R_P(y)||$.
		\item[(2)] The MK $\tau$-quantile contour is the set $Q_P(\mathcal{S}(\tau))$, and the MK depth region with probability content $\tau$ is $Q_P(\mathbb{S}(\tau))$.
		\item[(3)] The MK depth of $y \in \mathbb{R}^d$ with respect to $P$ is the depth of $R_P(y)$ under $D_P^{Tukey}$:
		\[
		\mathrm{D}_{P}^{\mathrm{MK}}(y):=\mathrm{D}_{U_{d}}^{\text {Tukey }}\left(\mathrm{R}_{P}(y)\right)
		\]
	\end{enumerate}
\end{definition}

\section{Empirical Depth, Ranks and Quantiles}
Now, we are ready to discuss the empirical version of MK quantiles, rankds and depth based on estimator $\hat{P}_n$ for $P$ and $\hat{F}_n$ for $F$.
\subsection{Empirical vector quantiles and ranks}
\begin{definition}
	The empirical vector quantile $\hat{Q}_n$ and vector rank $\hat{R}_n$ are any pair of functions satisfying, for each $u \in \mathcal{U}$ and $y \in \mathcal{Y}$ 
	\begin{IEEEeqnarray}{rCl}
		\hat{Q}_n &\in& \arg\sup\limits_{y \in \mathcal{Y}}\left[y^\top u - \hat{\psi}^*_n(y)\right], \nonumber \\
		\hat{R}_n &\in& \arg\sup\limits_{u \in \mathcal{U}}\left[y^\top u - \hat{\psi}_n(u)\right], \nonumber 
	\end{IEEEeqnarray}
where $\hat{\psi}^*_n(y)$ and $\hat{\psi}_n(u)$ is the empirical counterpart of (\ref{eq: psi}).
\end{definition}

Now, we introduce a Glivenko-Cantelli Theorem for empirical MK ranks and quantiles.

\begin{theorem}\label{eq: G-C}
	Suppose that the sets $\mathcal{U}$ and $\mathcal{Y}$ are two compact subsets of $\mathbb{R}^d$, and the probability measure $P$ and $F$ are absolutely continous with respective to Lebesgue measure, with $\spt(P) \subset \mathcal{Y}$ and $\spt(F) \subset \mathcal{U}$. Then, as $n \to \infty$, for any closed set $K \subset \mathrm{int}(\sup(P))$ and any closed set $K^\prime \subset \mathrm{int}(\sup(F)) $,
	\begin{IEEEeqnarray}{rCl}
		\sup\limits_{u \in K}||\hat{Q}_n(y) - Q_P(u)|| &\stackrel{P}\to& 0, \nonumber \\
		\sup\limits_{y \in K^\prime}||\hat{R}_n(y) - R_P(y)|| &\stackrel{P}\to& 0 \nonumber.
	\end{IEEEeqnarray}                                                                                                                    
\end{theorem}
\begin{remark}
	Recall that the Glivenko-Contelli class is usually insensitive to the continuous order-preserving transformation of the data, but this feature is not compatible with the moment assumption, since the existence of moment can not be guaranteed under such transformation. Thus, the compact support assumption in Theorem \ref{eq: G-C} is inappropriate.
	
	Moreover, the compact support assumption also makes the test procedures that based on this rank definition not exactly distribution-free, which leads to a restriction of usage. 
\end{remark}









