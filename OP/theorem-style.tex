%%%%%%%%% Theorem Environment

%%% Use mdframed

\usepackage[framemethod=TikZ]{mdframed}
\usepackage{amsthm}

\usepackage{fontawesome}
\usepackage{bbding}

% \mdfsetup{skipabove=\topskip,skipbelow=\topskip}
% \mdfdefinestyle{theoremstyle}{%
%   outerlinewidth=5pt,innerlinewidth=0pt,
%   outerlinecolor=gray!40,roundcorner=5pt
% }

%%% Proof

\renewcommand*{\proofname}{\blue{\textbf{Proof.}}}
% \renewcommand*{\qedsymbol}{$\mathwitch*$}
% \renewcommand*{\qedsymbol}{\faCoffee}
\renewcommand*{\qedsymbol}{\Peace}

%%% Theorem
\definecolor{Mingu-Red}{RGB}{163, 25, 25}
\mdfdefinestyle{theoremstyle}{%
    middlelinewidth=2pt,
    middlelinecolor=Mingu-Red,
    %%%%%%%%%
    linecolor=gray,linewidth=.5pt,%
    backgroundcolor=gray!10,
    %%%%%%%%%
    skipabove=8pt,
    skipbelow=5pt,
    %%%%%%%%%
    innertopmargin=.4\baselineskip,
    innerbottommargin=.4\baselineskip,
    innerleftmargin=7pt,
    innerrightmargin=7pt,
    %%%%%%%%%
    frametitlerule=true,%
    frametitlerulewidth=.5pt,
    frametitlebackgroundcolor=gray!15,
    frametitleaboveskip=0pt,
    frametitlebelowskip=0pt,
    %%%%%%
    % shadow=true,shadowsize=3pt,shadowcolor=black!20,
    theoremseparator={:},
    roundcorner=5pt,
}

\mdtheorem[style=theoremstyle]{exercise}{\faThumbTack\; Exercise}[section]
\mdtheorem[style=theoremstyle]{MDtheorem}{\faGraduationCap\; Theorem}[section]
\mdtheorem[style=theoremstyle]{proposition}{\faGraduationCap\; Proposition}[section]
\mdtheorem[style=theoremstyle]{corollary}{\faThumbTack\; Corollary}[section]
\mdtheorem[style=theoremstyle]{lemma}{\faThumbTack\; Lemma}[section]
\mdtheorem[style=theoremstyle]{problem}{\faQuestion\; Problem}[section]

%%% 定义theorem environment, 给theorem title加括号
\newcommand*{\Title}{}
\newenvironment{theorem}[1][]{%
    \ifstrempty{#1}{\begin{MDtheorem}}{\begin{MDtheorem}[(#1)]}%
}{%
    \end{MDtheorem}%
}%
%%%
\usepackage{lipsum}

\newtheoremstyle{examplestyle}% ⟨name⟩
{3pt}%⟨Space above⟩
{3pt}%⟨Space below⟩
{\upshape}%⟨Body font⟩
{}%⟨Indent amount⟩
{\bfseries}% ⟨Theorem head font⟩
{.}%⟨Punctuation after theorem head⟩
{.5em}%⟨Space after theorem head⟩2
{}%⟨Theorem head spec (can be left empty, meaning ‘normal’)⟩

\theoremstyle{examplestyle}
\newtheorem{myexam}{\HandPencilLeft\; Example}
\mdfsetup{%
    middlelinecolor=black,
    middlelinewidth=0.5pt,
    backgroundcolor=white,
    % roundcorner=5pt
}
\newenvironment{example}
  {\begin{mdframed}\begin{myexam}}
  {\end{myexam}\end{mdframed}}



\mdfdefinestyle{definitionstyle}{%
  middlelinewidth=2pt,
  middlelinecolor=Mingu-Red,
  %%%%%%%%%
  linecolor=gray,linewidth=.5pt,%
  backgroundcolor=white,
  %%%%%%%%%
  skipabove=8pt,
  skipbelow=5pt,
  %%%%%%%%%
  innertopmargin=.4\baselineskip,
  innerbottommargin=.4\baselineskip,
  innerleftmargin=7pt,
  innerrightmargin=7pt,
  %%%%%%%%%
  frametitlerule=true,%
  frametitlerulewidth=.5pt,
  frametitlebackgroundcolor=gray!15,
  frametitleaboveskip=0pt,
  frametitlebelowskip=0pt,
  %%%%%%
%   shadow=true,shadowsize=3pt,shadowcolor=black!20,
  theoremseparator={:},
  roundcorner=5pt
}

% \mdtheorem[style=lemmastyle]{remark}{$\blacklozenge$ Remark}[section]
\mdtheorem[style=definitionstyle]{definition}{\faKey\; Definition}[section]

\newtheoremstyle{remarkstyle}% ⟨name⟩
{3pt}%⟨Space above⟩
{3pt}%⟨Space below⟩
{\upshape}%⟨Body font⟩
{}%⟨Indent amount⟩
{\bfseries}% ⟨Theorem head font⟩
{.}%⟨Punctuation after theorem head⟩
{.5em}%⟨Space after theorem head⟩2
{}%⟨Theorem head spec (can be left empty, meaning ‘normal’)⟩

\theoremstyle{remarkstyle}
\newtheorem{remarkx}{\HandRight\; Remark}
\newenvironment{remark}
  {\pushQED{\qed}\renewcommand{\qedsymbol}{\PencilLeftDown}\remarkx} % 给 remark 环境定义结束符
  {\popQED\endremarkx}

% \definecolor{remarkcolor-bk}{RGB}{255,242,242}
% \definecolor{remarkcolor-line}{RGB}{0,242,12}
% \theoremstyle{remarkstyle}
% \newtheorem*{myremark}{Remark}
% \mdfsetup{%
%     middlelinecolor=remarkcolor-line,
%     middlelinewidth=1pt,
%     backgroundcolor=remarkcolor-bk,
%     roundcorner=5pt}
% \newenvironment{remark}
%   {\begin{mdframed}\begin{myremark}}
%   {\end{myremark}\end{mdframed}}



%%%%%%%%%%%%%%%%%%%%%%%%%%%%%%%%%%%%%%%%%%%%%%%%%%%%%%%%%%%%%%%%%%%%%%%%%%%%%%%%%%%%%%%%%%%%%%%%%%%% Use package amsthm

% \usepackage{amsthm}
% \theoremstyle{plain}
% \newtheorem{theo}{Theorem}
% \newtheorem{prop}{Proposition}
% \newtheorem{lem}{Lemma}
% \newtheorem{coro}{Corollary}

% \newtheoremstyle{definition}% ⟨name⟩
% {3pt}%⟨Space above⟩
% {3pt}%⟨Space below⟩
% {\itshape}%⟨Body font⟩
% {}%⟨Indent amount⟩
% {\bfseries}% ⟨Theorem head font⟩
% {.}%⟨Punctuation after theorem head⟩
% {.5em}%⟨Space after theorem head⟩2
% {}%⟨Theorem head spec (can be left empty, meaning ‘normal’)⟩

% \theoremstyle{definition}
% \newtheorem{exam}{Example}
% \newtheorem*{sol*}{Solution}
% \newtheorem{exer}{Exercise}
% \newtheorem*{ans*}{Answer}
% \newtheorem{defi}{Definition}

% \newtheoremstyle{note}% style name
% {2ex}% above space
% {2ex}% below space
% {\small\itshape}% body font
% {}% indent amount
% {\normalfont\bfseries}% head font
% {.}% post head punctuation
% {1ex}% post head space
% {}% head spec
% \theoremstyle{note}
% \newtheorem{remark}{Remark}
% \newtheorem*{hint*}{Hint}


